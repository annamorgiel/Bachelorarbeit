\section{Fazit}
Abschließend werden die Ergebnisse dieser Bachelorarbeit resümiert und ein Fazit
gezogen. Dazu wird zu den anfangs gestellten Forschungsfragen Bezug genommen und ein Ausblick auf mögliche Erweiterungen der E.V.A. App gegeben, welche im Rahmen von künftigen Abschlussarbeiten entstehen könnten.
\subsection{Zusammenfassung}
In dieser Arbeit wurde ein Prozess der Entstehung von einem Modul zum Training der Mimikry Fähigkeit vom Entwurf bis zur Implementierung beschrieben. Es wurde auch eine Abschätzung der Eignung des Moduls zu einem IT-gestützten Training der sozio-emotionalen Kompetenz untersucht, evaluiert und begründet. Diese Abschätzung der Eignung des Moduls ist aus einer Analyse der Usability und UX Studie entstanden. 
\subsection{Ausblick über die Forschungsfragen}
Es sollten Antworten auf diese Forschungsfragen gefunden werden, siehe Motivation, Kapitel 1:
\begin{enumerate}
    \item Welche Aspekte der Feinkonzeption des Mimikry-Moduls sind bei der Bewertung besonders positiv oder negativ ausgefallen?
    \item Basierend auf der ersten Forschungsfrage, wurde das Konzept des IT-gestützten Trainings zur Erweiterung der sozio-emotionalen Kompetenz durch die Stärkung der Mimikry Fähigkeit geeignet umgesetzt?
\end{enumerate}
Die Schwerpunkte der beiden Fragen beziehen sich auf die funktionelle Seite. Daher waren die Forschungsmethoden, die zur Evaluation dienen, aus diesem Bereich auszuwählen. Die Usability und UX Aspekte repräsentieren das Erlebnis der Interaktion mit der App. Dieses Spielerlebnis ist relevant für das Training, da sich die Benutzer nach einer positiven Erfahrung mit einem Spiel häufiger und lieber mit der App auseinandersetzen~\cite{Usability}. Das spielerische Erlebnis ist bei dieser Anwendung relevant, deswegen wurde das Konzept nach einem spielerischen und nutzerfreundlichen Design entwickelt.

\subsubsection{Rückblick auf die Ergebnisse der Studie}
Um die Ziele der Studie zu erreichen, wurden zwei quantitativen Forschungsmethoden zur Abschätzung von Usability und UX ausgewählt. Zusätzlich wurden durch die während der Studie entstandenen Beobachtungen und Anmerkungen der Benutzer protokolliert, um den qualitativen Aspekt der Studie zu verbessern.
Es wurde erkennbar, dass die App im Vergleich zum Benchmark bei folgenden Faktoren gut oder besonders gut abschneidet: Durchschaubarkeit, Stimulation, Originalität, Attraktivität und Effizienz. Am schlechtesten, jedoch über dem Durchschnitt, wurde Steuerbarkeit bewertet (siehe Abb.\ref{uxmimikry}). 
Die Ergebnisse wurden ausgewertet und analysiert. Dank der positiv ausgefallenen Auswertung lässt sich schließen, dass das Benutzererlebnis allgemein als positiv einzuschätzen ist. Es gab Stellen, die verbessert werden könnten, zum Beispiel sollte die Logik hinter dem zweiten Szenario erneut analysiert werden und ein direkteres Feedback für den Nutzer bezüglich der Kopfposition sollte vermittelt werden. Als größtes Problem stellte sich die Unsicherheit wegen des Mangels an Rückmeldung bezüglich der richtigen Positionierung des Tablets bei der zweiten Mimikry Variante heraus.

\subsubsection{Herausforderungen}
Eine der größten Herausforderungen war die Umsetzung vieler technischen Konzepte der Androidentwicklung. Die Umsetzung der MVP Architektur und das Speichern von Daten aus der Studie in einer lokalen Datenbank für die mögliche Fortsetzung im Rahmen einer Abschlussarbeit haben sich als besonders komplex herausgestellt. Die Hindernisse in diesem Fall lagen an der Anwendung von mehreren Android-spezifischen Konzepten, die trotz der bereits vorhandenen Android Vorkenntnissen einen Lern- und Implementierungsaufwand verursachten.

\subsubsection{Mögliche Weiterentwicklungen und Verbesserungen}. Folgende Stellen könnten verbessert werden:
\begin{enumerate}
    \item Die Logik hinter dem zweiten Szenario sollte restrukturiert werden. Dieses Spielszenario benötigt ein direktes Feedback für den Nutzer bezüglich der Kopfposition innerhalb/außerhalb dem Kamera Vorschau.
    \item Die Logik hinter dem Design sollte einheitlicher werden - Ein Beispiel hierfür wäre, den Screen mit der Vorstellung der Zielemotion einen Knopf hinzuzufügen.
    \item Die Logik hinter der Speicherung der höchsten Ergebnisse sollte aus den SharedPreferences zu den Klassen der lokalen Datenbank übertragen werden. Die Datenbank ist bereits vorhanden.
\end{enumerate} 

Weil der Gesamteindruck des Spiels allgemein als positiv von den Probanden bewertet wurde, stellt der Mangel der Umsetzung der Verbesserungsvorschläge kein entscheidendes Hindernis dar. Die oben genannten Stellen verletzen aber die Leitlinien, die Qualität einer guten Software kennzeichnen. 

Es sind weitere Implementierungen des IT-gestützten Trainings innerhalb der E.V.A. App möglich. Das Konzept des Mimikry Moduls wurde entwickelt und hat sich als gebrauchstauglich herausgestellt, die Wirksamkeit des computerbasierten Trainings wurde nicht untersucht. 

Drei weitere Mimicry Szenarien sind mit geringem Aufwand zu implementieren, basierend auf dem verfügbaren Code von guter Lesequalität mit einer übersichtlicher Dokumentation. Dazu könnte mit einem geringen Aufwand eine vollständige Studie durchgeführt werden, die neben der Gebrauchstauglichkeit auch die Wirksamkeit der Software Testen würde.

Die Daten, die während der Usability Studie gesammelt wurden und die vollständigen UX und Usability Studienergebnisse können auf eine Anfrage anonymisiert zur Verfügung gestellt werden. Im Appendix befinden sich Diagramme der Studienergebnisse, Beobachtungen die während der Studie gesammelt wurden, Bemerkungen von Probanden und Hintergrundinformationen zur Programmierung.

\subsubsection{Schlussfolgerungen}
Im Züge dieser Arbeit wurde ein Konzept des Mimikry Moduls entworfen und implementiert, also ein IT-gestütztes Training der sozio-emotionalen Kompetenzen. Das Training diente zur Stärkung der Mimikry Fähigkeit (siehe Kapiteln 1 und 2). Die Beschreibung der Konzeption befindet sich im Kapitel 3. Die relevanten Teilen der umfangreichen Implementierung befinden sich im Kapitel 4. 

Die Evaluierung der Eignung (nach der im Kapitel 1 genannten Definition handelt es sich nicht um die Wirksamkeit des Trainings, sondern um die Qualität des Spielerlebnisses) wurde durch Messung von UX und Usability realisiert und ist bei den beiden Fragebögen sehr positiv ausgefallen. Daher wurde das gesamte Spielerlebnis als hervorragend bewertet und die erhoffte Eignung wurde nach unserer Definition (Kapitel 1) zum größten Teil erreicht. 

Die möglichen Weiterentwicklungen im Rahmen einer Bachelor- oder Masterarbeit werden durch die gesammelten Daten und Ergebnisse ermöglicht. Daher lässt sich weiter schließen, dass die Feinkonzeption des Mimikry-Moduls erfolgreich konzipiert und implementiert wurde und das Konzept zum IT-gestützten Training der sozio-emotionalen Kompetenz durch Stärkung der Mimikry-Fähigkeit nach der in den Kapiteln 1 und 2 genannten Definition geeignet ist.
