\section{Dankesagung}
Zunächst möchte ich mich an dieser Stelle bei all denjenigen bedanken, die mich während der Anfertigung dieser Bachelorarbeit unterstützt und motiviert haben und an die, die zur Entstehung beigetragen haben, in dem sie mich beim Erwerb von Deutsch- und Informatikkenntnissen unterstützt haben. An der Stelle möchte ich noch mal betonen, dass die Arbeit sowohl selbständig implementiert und geschrieben wurde.


Ganz besonders gilt dieser Dank Herrn Dipl.-Inf. Tobias Moebert, der meine Arbeit betreute und über eine wertvolle Unterstützung während der praktischen und theoretischen Teil verfügte.
Dipl.-Inf. Dietmar Zoerner teilte sein tiefgründiges Verständnis der Autismus Problematik in dem Bezug auf Game Based Learning Aspekt mit mir.

Die Ausleihe der Gesichtserkennungssoftware von dem FI und erfolgte dank der Leiterin von dem Lehrstuhl „Komplexe Multimediale Anwendungsarchitekturen“ am Institut für Informatik und Computational Science der Universität Potsdam, Prof. Dr.-Ing. habil. Ulrike Lucke, die mich täglich seit dem Anfang meines beruflichen Weges als ein Vorbild verstärkte.
Ich möchte mich auch für die Unterstützung von dem Fraunhofer Institut bedanken, vor Allem bei Eugen Wagner und Sabine Stigler.

Ansonsten möchte ich mich bei allen meinen Freunden bedanken, vor allem bei Jerzy Michal Jurczyk, Mario Bodemann, Julia Kreise, Sven Koch, Maximilian Streubel, Peer Winkler, Roxana Tapia, Barbara Zimniewicz, Bailey, Blaubi, bei meiner Familie: Ewa Morgiel, Irena Sabaj und Stanislaw Sabaj, Franka Krasnicka und Janek Krasnicki. Auch bei vielen anderen, die ich wegen Platzmangen nicht erwähnen konnte.
Martin Freidank hat eine Open Source Latex Bibliothek entwickelt, mittels welcher diese Bachelorarbeit erstellt wurde~\cite{latex} die mit Hilfe von Christian Schulz-Hanke angepasst wurde. Ich werde es als ein Template für Studenten der Uni Potsdam zur Verfügung stellen.

Des Weiteren danke ich den Mitarbeiten vom Institut für Informatik an der Universität Potsdam und Freunden die an der Studie als Probanden beteiligt waren. Vor allem Julian Dehne hat mich während der Studienentwurfsphase durch die Erklärung der Methodik und Empfehlung von Quellen unterstützt.

Nicht zuletzt möchte ich mich bei dem Team von Women Techmakers Berlin bedanken, das auf einer ehrenamtlichen Basis unter Anderem Programmierer Anfänger unterstützt und mich während der Entstehung dieser Arbeit inspirierte: Natalie Pistunovich, Corina Gheorghe, Roxy Morris, Edward Medvedev, Miquel Beltran, Amanda Masuku und Karolina Kafel.

Nicht nur gaben sie mir immer wieder durch kritisches Hinterfragen wertvolle Hinweise – auch für die moralische Unterstützung und kontinuierliche Motivation haben sie mich einen großen Teil zur Vollendung dieser Arbeit beigetragen. Sie haben mich dazu gebracht, über meine Grenzen hinaus zu denken. 
Vielen Dank für eure Geduld und Mühen!

