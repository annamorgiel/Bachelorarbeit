\pagebreak
\thispagestyle{empty}
\section*{Abstrakt}
Diese Arbeit umfasst die Entwicklung, Umsetzung und Evaluation der Eignung des Konzeptes zum Training sozio-emotionaler Kompetenzen durch Stärkung der Mimikry-Fähigkeit. Dieses Training erfolgt durch eine Aufnahme des Gesichtsausdrucks, Emotionsanalyse der Software und eine direkte Reaktion des Benutzers auf das vermittelte Feedback. Die Implementierung erfolgte durch Einbindung einer Gesichtsanalysesoftware in eine bereits existierende Android App.

Das Training im Rahmen des Mimikry Moduls ermöglicht den Patienten mit ASS das Üben von verschiedenen Gesichtsausdrücken. Defizite im Bereich der Nachahmung von Emotionen werden für manche Personen aus der Gruppe sowohl als ein Hindernis in den sozialen Kontakten als auch im beruflichen Leben empfunden.

Die Umsetzung enthält sowohl eine Variante, in der Menschen ihre Ausdrücke sehen und eine wo es nicht vorhanden ist. Untersuchung der Eignung erfolgte durch eine Studie von Usability und User Experience. Die Wirksamkeit des Trainings wurde nicht untersucht, aber die während der Studienphase gesammelten Daten können für diesen und andere wissenschaftliche Zwecke freigegeben werden. 

Die Arbeit ist Interdisziplinär, deswegen enthält sie Erklärung der Grundkonzepte sowohl aus Psychologie als auch Informatik. Diese Arbeit ermöglicht eine Fortsetzung im Rahmen von einer Abschlussarbeit auch für Studierenden anderer Fachrichtungen als Informatik, da die Grundkenntnisse in Programmierung eingeführt wurden (siehe Kapitel 2 und Appendix).  