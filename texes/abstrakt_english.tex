\pagebreak
\thispagestyle{empty}
\section*{Abstract}
This work describes the design, development, and assessment of a comuter-based training of expressing emotions.
The training aims to improve the performance in social an professional interactions by exercising the mimicry, the ability to imitate others. 
The Mimicry Module is part of a game-based app E.V.A. (''Emotionen Verstehen und Ausdrücken''). 
The user's task is to mimicry the target emotion. The Software captivates the facial expressions, assesses the adequacy of the performed emotion using an external face recognition software, and gives direct, dynamic feedback to the user. Users adjust to the dynamically presented feedback.

Many patients with autism spectrum disorder(ASD) show deficits in understanding and showing emotions. Those deficits are perceived as an obstacle in both social and professional contacts. Mimicry Module enables an emotion training that is computer-based, so it does not rely on other people and is therefore most suitable for groups struggling with different forms of social anxieties. 

Two variants were implemented, one giving the preview of the user's face and one with no preview. The evaluation of Mimicry Module covered Usability and User Experience (UX). The effectiveness of the training has not been evaluated. 
During the Usability and UX study, the emotion-specific data have been collected and may be provided on demand. 
This also applies to the results of UX and Usability study. 

This work is interdisciplinary. The basic concepts from both psychology and programming were introduced in the second chapter and in the Appendix offering possible further continuations as a bachelor or a master thesis.
