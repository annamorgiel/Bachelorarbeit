\section{Motivation}
Diese Bachelorarbeit umfasst den Prozess der Entstehung eines IT-gestützten Trainings der sozio-emotionalen Kompetenzen. In dem ersten Kapitel werden die Grundidee hinter dem Training dieser Kompetenzen, die besondere Eignung für die Zielgruppe der Menschen mit einer Autismus-Diagnose und bisherigen Implementierungen eines solchen Trainings vorgestellt.

\subsection{Soziale und emotionale Kompetenzen im Bezug auf ASS}
Soziale und emotionale Kompetenzen spielen eine große Rolle in zwischenmenschlichen Beziehungen. Der richtige Umgang mit Emotionen hat einen entscheidenden Einfluss sowohl auf den mentalen Zustand als auch auf soziale Interaktionen~\cite{Pfeiffer.2012}. Emotionen werden als kurze, vorübergehende Gefühlszustände definiert. Sie entstehen unter anderem als Reaktion auf äußere Ereignisse. Emotionen beeinflussen das Denken und Handeln, daher haben sie einen entscheidenden Einfluss sowohl auf das psychologische Wohlbefinden, als auf die Qualität der zwischenmenschlichen Kontakte~\cite{Pfeiffer.2012}.

Einige soziale Gruppen, wie Menschen mit einer Autismus-Spektrum-Störung (ASS), weisen Defizite im Bereich sozio-emotionaler Kompetenzen auf. ASS gehört zu den tief greifendsten Entwicklungsstörungen und kann über die gesamte Lebensspanne auftreten~\cite{Spektrum2015}. Diese Entwicklungsstörung führt zu unterschiedlichen kognitiven Defiziten. Menschen auf dem Autismus-Spektrum fällt es häufig schwer, die Gefühle anderer zu erkennen~\cite{Hobson1993},~\cite{Dziobek.2008}. Das Spektrum ist sehr breit. Es werden unter anderem kognitiven Bereiche der Wahrnehmung, Einordnung und Wiedergabe der Gefühle beeinträchtigt~\cite{Kliemann.2012}\textsuperscript{,}~\cite{South.2011}.

Menschen mit ASS, vor allem Kinder, zeigen häufig eine gewisse Technikaffinität und verbringen gerne ihre Freizeit in der digitalen Welt~\cite{Moore2000}. Ein computerbasiertes Training sozio-emozionaler Kompetenzen entspricht daher besonders gut ihren Interessen und Bedürfnissen~\cite{Kohls.2011}.

Soziale und Emotionale Kompetenzen sind eng mit der Wahrnehmung und Wiedergabe der Emotionen verbunden. Emotionale Kompetenz bezeichnet Aspekte des Lebens, die mit Emotionen gekoppelt sind. Zu den wichtigsten Merkmalen der emotionalen Kompetenz zählen Wiedergabe des eigenen mimischen Emotionsausdrucks, Erkennen des mimischen Emotionsausdrucks anderer Personen, sprachlicher Emotionsausdruck, sowie Emotionswissen und -verständnis und auch Emotionsregulation~\cite{Pfeiffer.2012}. Soziale Kompetenz bezeichnet die Fähigkeit, persönliche Ziele in sozialen Interaktionen zu erreichen und positive Beziehungen über die Zeit und über verschiedene Situationen aufrecht zu erhalten~\cite{Pfeiffer.2012}.Die Qualität des Umgangs mit Emotionen beeinflusst soziale Interaktionen im Wesentlichen.
Ein beidseitiges Verständnis in der Kommunikation erfolgt nicht nur auf der sprachlichen Ebene, sondern auch auf der Ebene der Körpersprache, unter anderem durch die Mimik.
Im Rahmen dieser Arbeit wird ein IT-gestütztes Training der sozio-emotionaler Kompetenz der Mimik, also der Emotionswiedergabe, implementiert und evaluiert.

\subsection{Training der Kompensationsmechanismen\\ in den verwandten Arbeiten}
Der neueste Stand der Forschung zeigt vielversprechende Ergebnisse beim Ausgleich der Defizite durch ein spiel-basiertes Trainieren der verschiedenen Kompensationsmechanismen. In den verwandten Arbeiten wurde (oder wird gerade im Rahmen einer langjährigen Studie) ein Training der sozio-emotionalen Kompetenz untersucht. In diesen Arbeiten werden Erkenntnisse und Ansätze aus dem E-Learning Forschungsfeld und auch aus der Emphatieforschung berücksichtigt~\cite{Zirkus}. 
Wissenschaftler und Firmen entwickeln verschiedene computerbasierte Trainings für Emotionserkennung (z.B. FASTER~\cite{Faster} und SCOTT~\cite{Scott}).
Einige dieser Trainingsprogramme zeigen durchaus positive Effekte. Doch im Alltag profitieren die Betroffenen nur wenig von diesen, da sie die neu erlernten Fähigkeiten nicht effektiv übertragen und einsetzen können. Daher werden immer weiter intensiv neue Ansätze erforscht.
Im Rahmen vom Forschungsprojekt SCOTT wurden zwei computerbasierte Trainings entwickelt: SCOTT für Erwachsene und Zirkus Emphatico für Kinder. Ein Training der sozio-emotionalen Kompetenz durch die beiden Programme stärkt die emotionale Empathie, grundlegende Verarbeitung und Verbalisierung eigener Emotionen. Zudem wird eine Übertragbarkeit in den Alltag trainiert. 
Durch Üben mit der SCOTT Software werden Gesichsausdrücke, Stimmen und Situationen mit sozialen Kontext verknüpft. Das Training bei ''Zirkus Empathico'' erfolgte durch wiederholtes identifizieren und benennen der Gesichtsausdrücke mit verschieden Emotionen~\cite{Scott}\textsuperscript{,}~\cite{Zirkus}.
Durch die für die Kinder im Alter von 5-10 Jahren geeignete, computergestützte Software ''Zirkus Empathico'' wurde eine Verbesserung der sozialen Kompetenzen beobachtet~\cite{Zirkus}.
Der Aspekt von Erkennung der Emotionen wurde aktiv geübt und die Fähigkeit, Emotionen auszudrücken, wurde in Form eines Generalisierungmoduls zur Repräsentation der Situationen aus dem Alltag umgesetzt und war im Bezug auf das gesamte Spiel einzusehen. 

Als eine Erweiterung der vorhandenen Trainingssoftware wurde aus SCOTT ein Ansatz des Mimikry Moduls entworfen. Die Software dient im Unterschied zu den verwandten Projekten hauptsächlich dem Zweck der Stärkung der Emotionswiedergabe.

\subsubsection{IT-gestützes Training im Projekt EMOTISK}
Das Projekt EMOTISK ist ein interdisziplinäres Projekt, das an dem Lehrstuhl Komplexe Multimediale Anwendungsarchitekturen an der Universität Potsdam unter Beaufsichtigung von Prof. Dr.-Ing. habil. Ulrike Lucke entwickelt wird. 
Im Mittelpunkt der Forschung steht DIE Unterstützung von Menschen, deren sozial-kognitive Fähigkeiten nicht vollständig entwickelt sind oder auf andere Art und Weise beeinträchtigt wurden. Die Ansätze und Minispiele der Trainingssoftware SCOTT werden im Rahmen von EMOTISK weiterentwickelt, in dem die in SCOTT vorhandenen Stimuli in neue Minispiele integriert werden~\cite{Zoerner.2017}.

\subsubsection{E.V.A. - Emotionen Verstehen und Ausdrücken}
E.V.A. ist eine spiel-basierte Android App. Sie wurde als Erweiterung der spiel-basierten Trainingssoftware SCOTT im Rahmen von Projekt EMOTISK~\cite{Scott} entwickelt. Um den kognitiven Kapazitäten der Zielgruppe gerecht zu werden, wurden einige Entwurfsentscheidungen vorgenommen.

Die App wurde unter Beachtung der GBL und der für die spezifische Zielgruppe angepassten Prinzipien und Lösungen entwickelt. Eine der umgesetzten Lösungen bestand aus Implementierung eines adaptiven Bewertungssystems, das eine Verbesserung des Spielergebnisses ermöglichen sollte. Das Niveau eines Spielers wird ständig aus seinen Spielergebnissen geschätzt. Es werden zu dem entsprechenden Schwierigkeitsgrad passende Aufgaben generiert.
Die App wurde auch in einem unterstützenden, fröhlichen Design entwickelt. Diese Aspekte wurden berücksichtigt, um den Frustrationsfaktor zu minimieren und das Spielerlebnis zu verbessern.

Das E.V.A. Spiel besteht aus mehreren Modulen, unter anderem Gesichterpuzzle, Stimmenpuzzle, Filmpuzzle und Emoblitz. Je nach Art der Übung lernt der Benutzer in einer visuellen oder auditiven Form eine Zielemotion zu erkennen oder zu identifizieren. Das lässt sich auf dem Beispiel vom Gesichterpuzzle erkären. Dieses Modul hat zwei mögliche Variente an Aufgaben. Wenn eine Aufgabe aus einer Zuordnung von passenden Paaren der Augen- und Mundausdrücke besteht, siehe \ref{facepuzzle_implicite}, wird der Prozess als eine implizite Emotionserkennung oder Erkennung bezeichnet. Wenn eine Emotion als Wort aus einem Set der verfügbaren Möglichkeiten gewählt werden sollte, findet ein Prozess der expliziten Emotionserkennung - Identifizierung statt, vgl.\ref{facepuzzle_explicite}.

\subsection{Zielsetzung}
\subsubsection{Anwendungsdomäne}
IT-gestütztes Training sozio-emotionaler Kompetenz besitzt einige praktischen Anwendungsdomänen – im Mittelpunkt der Forschung steht der positive Einfluss des Trainings auf die Patienten mit einer Autismus Diagnose, aber auch ältere Menschen oder Menschen aus den anderen Kulturkreisen könnten davon profitieren. Durch die Benutzung der App fühlen sie sich wohler bei zwischenmenschlichen Kontakten. 
Die Patienten mit Autismus werden aber besonders relevant, weil die vorherigen Studien in dieser Gruppe nachgewiesen haben, dass die Menschen mit ASS in der Lage sind, Kompensationsmechanismen zu trainieren. Durch dieses Training der sozialen und emotionalen Kompetenzen werden die Defizite in der Wahrnehmung und Einordnung der Gefühle ausgeglichen~\cite{Zoerner.2017}. 

Die Studien sind noch nicht vollständig durchgeführt worden, aber die gesammelten Beweise weisen eindeutig darauf hin, dass die Patienten zwar sich im Durchschnitt verbessert haben, was das Erkennen und Identifizieren der Emotionen angeht, es war aber für sie allgemein schwierig, die erworbenen Kompetenzen auf das Alltagsleben zu übertragen. Diese Übertragbarkeit der erworbenen Kompetenzen kann auch gezielter trainiert werden. Aus dieser Überlegung ist die Idee des Mimikry Moduls entstanden und es wurde für die Umsetzung im Rahmen einer Bachelorarbeit vorgeschlagen.

\subsubsection{Feinkonzeption, Umsetzung und Evaluation des Mimikry Moduls}
Das Mimikry Modul ist eine Erweiterung der E.V.A. App, die im Rahmen dieser Arbeit entstanden ist. Während der Entwicklung von E.V.A. wurden Grafiken erstellt, die einen groben Ablauf vermittelt haben, jedoch weder fester Abluaf noch Code wurde erstellt. Diese Aufgaben wurden in dieser Bachelorarbeit realisiert. Eine Face Recognition Software wird bei dem Modul verwendet, um die Generalisierung von dem IT gestützten Training zu verbessern. \\\\
Es gibt insgesamt fünf Szenarien die mit externer Kooperation entwickelt und im Form von Grafiken zur Verfügung gestellt wurden. Diese Grafiken haben einen groben Ablauf der fünf Szenarien dargestellt. Zwei der fünf  Szenarien wurden im Details konzipiert und umgesetzt, einer Szenario mit einer Kamera Vorschau und eine ohne Vorschau. Der Benutzer wird beim ersten Szenario in der Lage sein, sich selbst zu sehen. Bei dem zweiten Szenario wird ihm dies nicht ermöglicht. Die detaillierte Beschreibung befindet sich im Kapitel 3, Konzeption.
Es wurde im Rahmen der Evaluation die Eignung des IT-gestützten Trainings und nicht die Wirksamkeit oder Therapiefähigkeit untersucht.

\paragraph{Die Eignung des IT-gestützten Trainings der sozio-emotionalen\\ Kompetenz durch Stärkung der Mimikry Fähigkeit}wird im Rahmen dieses Moduls evaluiert. Der Begriff Eignung repräsentiert ein Spektrum der Variablen, die das eigentliche Training beeinflussen könnten wie zum Beispiel das ungestörte Spielen ohne auftretenden Fehlermeldungen. Die Aufgaben aus diesem Training, also aus dem Mimikry Modul, sollten regelmäßig geübt werden. Das Eignung der Software zum Üben der Emotionen kann auf verschiedenen Ebenen gemessen werden. In dem Kapitel Evaluierung wird der Prozess des Entwurfs einer Studie beschrieben, die sich mit Erfahrungen von Benutzern auseinandergesetzt hatte. 
\newpage
\subsection{Aufbau dieser Arbeit}
Im zweiten Kapitel werden die Grundlagen für diese Arbeit erörtert. Dazu gehören sowohl technologischen als auch psychologischen Grundlagen, relevante Terminologie und Konzepte.
Das dritte Kapitel umfasst eine Feinkonziepierung des Mimikry Moduls, eine detaillierte Beschreibung der ausgewählten Szenarien und die Begründung der getroffenen Entscheidungen.
Das vierte Kapitel erklärt die Umsetzung der auf der Model-View-Presenter basierender Architektur, die relevanten Implementierung Entscheidungen und die Anwendung der Room Bibliothek für Verwaltung einer lokalen Datenbank.
Das fünfte Kapitel schildert die Vorbereitung, Durchführung und Auswertung einer abschließenden Usability und UX Studie, in welcher das Mimikry Modul ausführlich getestet, evaluiert und analysiert wurde. 
Die Arbeit schließt mit einem Fazit und Ausblick in einzelne Kapitel.
